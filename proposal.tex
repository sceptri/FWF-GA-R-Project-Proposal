\documentclass[a4paper,11pt]{scrartcl}
\usepackage{proposal}
\usepackage[usenames,dvipsnames,table]{xcolor}
\usepackage{enumitem}
\usepackage{hyperref}
\addbibresource{proposal.bib}

\DeclareAcronym{UFO}{short=UFO, long=ultra-fast oscillation} %KB, changed to imaging, MU

% Both acronym and title are subject to change
\sfbacro{EpilepsyOptimization}
\subject{Proposal for International Principal Investigator Project}
\title{Optimization and identification of epilepsy-related dynamical systems with respect to signal features}

\begin{document}

\maketitle
\tableofcontents

\newpage

\section{Current state of research}\label{sec:state}

% Says nothing, thus I don't think it'd be a great idea to include...
% In nature, optimization can be seen as a fundamental principle underlying many diverse mechanisms, from plants orienting themselves toward light as they grow to ants discovering the most efficient paths to resources. It is reasonable, then, to assume that certain observed phenomena are direct outcomes of these underlying optimization processes. This perspective raises an intriguing question: how might applying the concept of optimization to current open problems in science enhance our understanding of them?

% Applied mathematics and neuroscience have a long history of complementing each other in the quest to understand the intricacies of the brain. Mathematical models provide a structured way to analyze the dynamics of neuronal systems, offering insights that are difficult to achieve through experimental observation alone. In neuroscience, these models can be used to shed light on various topics such as memory encoding \cite{Rolls1996, Guzman2016}, or information processing, and the generation of rhythmic brain activity \cite{Hoppensteadt1997, Izhikevich2006, Song2018}, which are closely connected to complex synchronization patterns in neuronal networks. 
% TODO: Maybe add applications to other disciplines

Complex synchronization patterns in neuronal networks are linked to key capabilities of brain, such as memory encoding, information processing, or the generation rhythmic brain activity \cite{Izhikevich2006, Song2018}. Abnormalities in rhythmic brain activity can be attributed to neurological disorders, for example, epilepsy. From the point of view of neuronal dynamics, these abnormalities manifest themselves as anomalous synchronization within neuronal networks and can lead to spontaneous seizures \cite{Jiruska2013}. Abnormal synchronization is often close tied with the presence of high-frequency oscillations (HFOs), very high-frequency oscillations (VFHOs), and \acp{UFO} within depth electroencephalographic (EEG) recordings from patients with focal epilepsy. This presumed connection has fueled the research interest due to the potential use as biomarkers for epileptogenic zones \cite{Jacobs2008, Worrell2011, Staba2011, Jacobs2012, Zijlmans2012, Brazdil2017, Cimbalnik2018, Cimbalnik2020, Brazdil2023}. Nonetheless, taking into account the physiological limitations of action potential firing rates \cite{Gabbiani2010}, the exact governing mechanism for the generation of such high-frequency signals in neuronal networks remains a mystery \cite{Jiruska2017}.

Existing research endeavors provide insight on how anti-phase synchronization can serve as a potential mechanism for the emergence of HFOs and VHFOs in epileptic EEG signals \cite{Pribylova2024, Sevcik2024, Zathurecky2025}. Specifically, these analyses, based mostly on techniques of bifurcation theory, theoretically examined small numbers of coupled oscillators and provided numerical computations confirming the findings on a large variety of conductance-based neuron models and scenarios (including stochastic simulations of neuronal populations).
% TODO: Add citations to other groups (if relevant)

Crucially, the exact relationship between different granularities of neuron clusters is still unknown. More precisely, it remains an open problem how dynamics of intracellular activity of individual conductance-based neuron models, or even coupled neuronal networks consisting of several neurons, influences the local field potential (LFP) of the entire epileptogenic cluster. While synchronization between neurons (or their networks) is hypothesized to be a crucial mechanism behind epileptogenic activity, it remains to be seen how exactly is this achieved in the full-scale neuronal cluster and how it propagates to its LFP.

On the side of optimization and optimal control theory (OCT), models of neuronal populations \cite{Salfenmoser2022, Salfenmoser2024} and their networks \cite{Chouzouris2021} have been the subject of open-loop optimal control techniques. Phase reduction \cite{Zlotnik2012, Dasanayake2011, Pietras2019} techniques were used for control of oscillatory phenomena exhibited by neuronal systems, dramatically simplifying the complex dynamics along stable limit cycles by a phase-only parametrization, allowing even for the study of large-scale networks \cite{Bomela2023}. Moreover, OCT has been used for control of oscillations and network synchrony in neuronal models \cite{Ruths2014, Orieux2024, Popovych2006, Costa2024}.
% optimal control theory (OCT) has proven to be invaluable to control oscillations and network synchrony , as it provides tools for computation of efficient stimulation for linear or non-linear systems \cite{Kirk2004}. In particular, control of neuronal systems has been a target of intensive research \cite{Kao2019, Suppa2016, Liu2018} with closed-loop control and machine learning methods being put the task of optimizing stimulation protocols for the treatment of neurological disorders \cite{Yu2020} and modulating brain activity \cite{Tafazoli2020, Park2019}. Furthermore, models of neuronal populations \cite{Salfenmoser2022, Salfenmoser2024} and their networks \cite{Chouzouris2021} have been the subject of open-loop optimal control techniques. Phase reduction \cite{Zlotnik2012, Dasanayake2011, Pietras2019} techniques were used for control of oscillatory phenomena exhibited by neuronal systems, dramatically simplifying the complex dynamics along stable limit cycles by a phase-only parametrization, allowing even for the study of large-scale networks \cite{Bomela2023}. As a special case, we note that optimal control of oscillators can provide us with tools for a rapid control of systems in time-sensitive manner \cite{Kamzolkin2024, Santaniello2011}.

Neuroscience primarily studies complex networks of a large number of intricately coupled neurons, where emergent properties play a key role in governing the dynamics. As such, system identification has been explored as a tool for aiding understanding of systems' structure, be it for identification of the systems as a whole \cite{Brunton2016}, the discovery of the network configurations \cite{Owens2022} or learning missing terms in dynamical systems undergoing bifurcations \cite{VortmeyerKley2021}. A large variety of approaches have been proposed to deal with input noise, robustness \cite{Kaheman2020, Rosafalco2024}, or utilization of problem-specific properties \cite{Bisheban2017, Jongeneel2022}. Additionally, system identification has been used to enhance other mathematical methods such as optimal control \cite{Morrison2021}, providing invaluable insights into the structure of the systems, which can then be exploited.

Also, there has been an active area of research concerning synchronizability of networks depending on their configuration \cite{Hong2004}. Several factors influencing synchronization capabilities of networks have been proposed and studied with substantial efforts going into exploring this phenomenon on small-world networks \cite{Nishikawa2003}. Kuramoto model of synchronization has been the subject of inquiries into optimal networking design targeted at synchronization \cite{Fazlyab2017, Nishikawa2006} and optimal control, which has been used to replicate measured functional relationships from cortical oscillations in human brain \cite{Menara2022}.

\section{Aims of the project}\label{sec:aims}

% Our principal desire for this project is to connect the bifurcation theory and the optimization viewpoints to neuroscience with special emphasis on focal epilepsy. This connection should reveal currently hidden insights into the governing principles of dynamics of neuronal networks.

This project focuses on advancing the mathematical modeling of focal epilepsy, with an emphasis on identifying the structure and topology of neuron clusters that likely encompass epileptogenic zones. This is achieved through the analysis of multi-scale intracranial EEG measurements obtained from specialized electrodes. To examine relationships between measurements at varying granularities, relevant signal features will be defined, providing corresponding loss functions for employment of data-driven and optimization methods.

We strive to set up a mathematically rigorous notion of various signal features (e.g. frequency bands for spiking or bursting, interburst intervals, VHFOs, and \acp{UFO}, etc.) \cite{Wedler2022} to be used in optimization or epileptic event detection, etc. Modification of current optimization and data-driven methodologies to accommodate for selected signal features is planned. Crucially, such construction should allow for definition and selection of appropriate properties of epileptic EEG signals at different granularities, enabling quantifiable comparison not only between each granularity level but also with simulated data.

In parallel, we aim to utilize the techniques of optimal network design and data-driven methods \cite{Baggio2021} to achieve synchronous dynamics with characteristics related to the epileptogenic zone (under physiological constraints). As such, this should closely align with modeling of LFP using coupled networks, and performing certain optimization procedures with respect to coupling strengths or delays between oscillators. 

Our third goal lies mostly in system identification, where we wish to study derivative-free methods \cite{Bortz2023, Messenger2024}. Specifically, we will attempt identification of a network configuration from \textit{in vivo}, \textit{in vitro} and \textit{in silico} experimental data. This could provide a concrete neuronal cluster topology and, ideally, improve the localization of the epileptogenic zone based on interictal biomarkers (HFOs, VHFOs and \acp{UFO}). Importantly, the identification task should also be focused on discovering models or networks whose solution do not necessarily fit the experiment data exactly, but contain the same signal features.

\section{Innovation}\label{sec:innovation}

We perceive the key contributions to a broader scientific community in several areas. Firstly, the rigorous treatment of signal features should be relevant and highly useful not only in the field of neuroscience, but also among other fields where certain properties of an output signal of a controlled system are desired. In conjunction, development of optimization and data-driven methods affected or constrained by requirements on input or output signal properties should make this framework versatile. Applications for a given system might include detection of events, modeling thereof, and optimal control such that certain requirements on the features of the output signal are fulfilled.

Nevertheless, primary innovation of the proposed project lies in the application of optimization, data-driven methods and the notion of signal features to extract insight about the structure and topology of epileptogenic neuronal clusters and how they behave on different spatial and measurement scales. This should shed more light on HFOs, VHFOs and \acp{UFO} and their role as biomarkers of epilepsy. Furthermore, these structural insights will likely be possible to use to improve focal epilepsy biomarkers, more accurately pinpointing the exact epileptogenic area.

% Primary innovation of the proposed project lies in the application of the notion of signal features specifically to mathematical treatment of HFOs, VHFOs and even \acp{UFO}. We believe such approach can greatly enhance current modeling capabilities in this niche, providing sound explanation for the emergence of non-physiological frequencies in local mean field potential measurements and their importance as biomarkers. Moreover, studying ways of controlling the anomalous oscillations may likely prove fruitful in their suppression. By further exploring system identification from the complex EEG signals, possibly also constrained by certain signal features, it is plausible we will be able to exploit discovered structures in our benefit.

% TODO: Maybe, add something about SINDyOC and derivative-free equation discovery methods (and OCT <-> eq. discovery relation)

\section{Methods}\label{sec:methods}

% TODO: Find relevant citations (and be more specific) 
For this project, we are considering two distinct ways of defining signal features -- wavelet theory and topological data analysis. Signal features defined via wavelet theory rely on choosing a reference function, which should ideally capture the essence of our signal feature, and a similarity metric (e.g. inner product in Hilbert space) \cite{Ganjalizadeh2022, Brochard2022, Rocha2011}. The benefit of the approach is the innate inclusion of the notion translations and scaling \cite{Mallat2009} -- two very important properties any reasonable signal feature definition must allow. Definition based on topological data analysis, namely topological invariants or persistent homologies \cite{Carriere2021, Myers2022}, requires far more theoretical background, but it should possibly allow for expression of a wider collection of signal features in mathematically rigorous way. Moreover, either of these two methods can likely succeed each other and thus mitigate the risk of either being unsuitable.

Let us also note the EEG data \cite{Brazdil2017} were measured using hippocampal depth electrode implanted within the presurgical investigation of patients suffering from drug-resistant focal epilepsy. The micro-electrodes and macro-electrodes were spatially distributed, and the measured neural mass activities at different places were interconnected in the brain. This introduces the need to study and analyze the dynamics of the subsystems (subpopulations) of neural mass with various interconnections.

The seminal study of the relations between various granularities of the epileptogenic neuronal cluster is to be executed in an iterative fashion. For each iteration of this process, we will primarily specify relevant signal features (at each granularity) and constraints on network topology. With these assumptions, we will test whether we can explain the measured relationship between dynamics on micro- and macro-electrodes. Our secondary goal will be to identify a fitting network such that output signal of the model at each granularity is similar to the measured one.

As one relaxes the constraints on possible network topologies, the search for a fitting one becomes all the harder, necessitating the need for novel optimization and data-driven methods, such as combinatorial optimization \cite{Dai2011}, robust optimization \cite{Ben-Tal2009}, geometric optimal control \cite{Agrachev2004}, turnpike property exploitation \cite{Gugat2024} or convex relaxations \cite{Eltved2021}. Careful consideration of numerical optimization methods, for example, (adaptive) Barzilai-Borwein \cite{Zhou2024}, may prove necessary.

% We aim to perform optimal control using standard tools of calculus variations, like the Lagrange multiplier approach \cite{Kunisch2008}. Nonetheless, for the high dimensional case of optimal control of a neuronal network (with each node of the network being possibly a relatively high dimensional model of neuron itself), we may need to turn to more novel methods, such as geometric optimal control \cite{Jurdjevic1996, Agrachev2004}, taking advantage of the topology of the network \cite{Gao2022, Nishikawa2006, Gueant2019}, or exploiting the turnpike property \cite{Zaslavski2023, Gugat2024}.

% In terms of mathematical programming, convex optimization \cite{Boyd2004} and corresponding relaxations \cite{Eltved2021} seem favorable. For practical purposes, considerations regarding robust optimization \cite{Ben-Tal2009} will likely be necessary, as well as careful selection of numerical optimization methods, for example, (adaptive) Barzilai-Borwein method \cite{Zhou2024}. Network topology optimization primarily relies on combinatorial optimization \cite{Korte2005, Dai2011, Osmolovskii2023}, discrete calculus \cite{Grady2010}, and graph theory results.

\section{Work plan and timeline}\label{sec:workplan}

% Specific goals:
% - study of relationships with respect to granularity of local field potential and its relation to smaller-scale models (up to individual neurons)
% - definition of signal features and characterstics such that the above study is even possible
%   - and for determination whether numerical simulations behave the same on this multi-scale scenario
% - data-driven methods/optimization to find suitable network topologies such that numerical simulations with discovered networks mimic the real data (as viewed by their features/properties) on all granularities

% Project description must include:
% - The distribution of work between the partners
% - The added value of the international co-operation

We divide our proposed project into 5 tasks spread out over the three year duration, see Table \ref{tab:Gantt} for a Gannt chart of planned progress:

% 1. definition of signal features and development of event-detection methodology
% 2. 
% 3. exploration and validation of measured dynamics at each granularity with respect to selected signal features
% 4. constrained network identification from real data and simulational validation

\begin{itemize}
    \item[\textbf{Task 1.}] \textit{(blue)} Definition of key signal features (for example, frequency bands for spiking or bursting, VHFOs or \acp{UFO}, interburst intervals) via wavelet theory or topologically. Development of event detection techniques based on chosen methodology for verification.
    
    \item[\textbf{Task 2.}] \textit{(purple)} Construction of loss functions corresponding to selected signal features for optimization and data-driven methods. Building analytical and numerical methods for solving proposed problems, including, but not limited to, considerations of discretization, computational complexity, and robustness. Proof-of-concept implementation and numerical experiments.

    \item[\textbf{Task 3.}] \textit{(green)} Modeling of LFP, as measured \cite{Brazdil2017} on macro-electrodes, and its relation of micro-electrodes' readings and individual conductance-based intraceullar neuron models. Proposition of viable network topology families (i.e. prescribing given possible network connectivities) with respect to coupling strength and delays. Validation whether said topology can explain measured data.
    
    \item[\textbf{Task 4.}] \textit{(orange)} Study of (derivative-free) data-driven system discovery methods, possibly based on \textit{sparse identification of nonlinear dynamics} (SINDy) \cite{Brunton2016}, in collaboration with Dr. Sébastien Court from Innsbruck. Consideration of both theoretical analysis and numerical schemes, along with development of suitable numerical experiments. Application of system identification methods to data on individual neurons.

    \item[\textbf{Task 5.}] \textit{(yellow)} Identification of a fitting neuronal network based on selected signal features and constraints on topology of the network from \textit{in vivo}, \textit{in vitro}, and \textit{in silico} data. Sensitivity analysis with respect to connection and/or neuron (network) removal.

    % \item[\textbf{Task 5.}] \textit{(yellow)} Parameter and network topology identification from \textit{in vivo} (see \textit{Task 1}), \textit{in vitro}, and \textit{in silico} data. Both theoretical analysis and numerical schemes will be considered, along with development of suitable numerical experiments. Our goal will be to discover models and networks, whose solution do not necessarily fit the experiment data exactly, but contain the same signal features.
\end{itemize}

We aim to submit a co-authored paper (mP1) on derivative-free system identification, inspired by SINDy, in collaboration with Dr. Sébastien Court from the University of Innsbruck. Additionally, we plan to apply event detection methods based on our signal features to real-world EEG measurements, contributing to a data analysis paper submission (mP2) in partnership mainly with Dr. Veronika Eclerová, Masaryk University. On a similar note, we expect to submit a publication on the framework of signal feature definition and subsequent optimization together with the results our methodology will yield in regards to LFP and the structure of epileptogenic neuronal networks (mP3).

% TODO: Add anything non-method related (any visits/conferences?)

\setlength\extrarowheight{3pt}
\begin{table}[h!]
    \centering
     \addtolength{\tabcolsep}{-3pt}
     \fontsize{10pt}{11pt}\selectfont
    \begin{tabular}{|c|c|c|c|c|c|c|c|c|c|c|c|c|}
    \hline
    \multicolumn{13}{|c|}{\cellcolor{gray!50} Year 1} \\
        \hline
        Work plan item & 1 & 2 & 3 & 4 & 5 & 6 & 7 & 8 & 9 & 10 & 11 & 12 \\
        \hline
        Derivative-free SINDy & \cellcolor{orange!25} & \cellcolor{orange!25} & \cellcolor{orange!25} & \cellcolor{orange!25} & \cellcolor{orange!25} & \cellcolor{orange!25} & \cellcolor{orange!25} & \cellcolor{orange!25} mP1 & & & & \\
        \hline
        Signal feature definition & & \cellcolor{blue!25} & \cellcolor{blue!25} & \cellcolor{blue!25} & & & & \cellcolor{blue!25} & \cellcolor{blue!25} & \cellcolor{blue!25} & \cellcolor{blue!25} & \\
        \hline
        Event detection & & & & \cellcolor{blue!25} & \cellcolor{blue!25} & \cellcolor{blue!25} & & & & \cellcolor{blue!25} & \cellcolor{blue!25} & \cellcolor{blue!25} \\
        \hline
        Event signal analysis & & & & & & \cellcolor{blue!25} & \cellcolor{blue!25} & \cellcolor{blue!25} & & & & \cellcolor{blue!25} \\
        \hline
        Loss functions & & & & & \cellcolor{purple!25} & \cellcolor{purple!25} & & & & & \cellcolor{purple!25} & \cellcolor{purple!25} \\
        
    \hline
    \multicolumn{13}{|c|}{\cellcolor{gray!50}Year 2} \\
        \hline
        Work plan item & 13 & 14 & 15 & 16 & 17 & 18 & 19 & 20 & 21 & 22 & 23 & 24\\
        \hline
        Event signal analysis & \cellcolor{blue!25} & \cellcolor{blue!25} & \cellcolor{blue!25} & \cellcolor{blue!25} & \cellcolor{blue!25} mP2 & & & & & & & \\
        \hline
        Loss functions & & \cellcolor{purple!25} & \cellcolor{purple!25} & & & & & & & & & \\
        \hline
        Analytical study & & & & \cellcolor{purple!25} & \cellcolor{purple!25} & \cellcolor{purple!25} & \cellcolor{purple!25} &  & & & & \\
        \hline
        Numerical methods & & & & & & \cellcolor{purple!25} & \cellcolor{purple!25} & \cellcolor{purple!25} & \cellcolor{purple!25} & & & \\
        \hline
        Proof-of-concept & & & & & & & & \cellcolor{purple!25} & \cellcolor{purple!25} & \cellcolor{purple!25} & \cellcolor{purple!25} & \\
        \hline
        Features \& topology selection & & & & \cellcolor{green!25} & \cellcolor{green!25} & \cellcolor{green!25} & & & & \cellcolor{green!25} & \cellcolor{green!25} & \cellcolor{green!25} \\
        \hline
        Validation by \textit{in vivo} data & & & & & & & \cellcolor{green!25} & \cellcolor{green!25} & \cellcolor{green!25} & & & \\
    \hline  
    \multicolumn{13}{|c|}{\cellcolor{gray!50}Year 3} \\
        \hline
        Work plan item & 25 & 26 & 27 & 28 & 29 & 30 & 31 & 32 & 33 & 34 & 35 & 36 \\
        \hline
        Features \& topology selection & & & & & \cellcolor{green!25} & \cellcolor{green!25} & & \cellcolor{green!25} & & & & \\
        \hline
        Validation by \textit{in vivo} data & \cellcolor{green!25} &  \cellcolor{green!25} & & & & & \cellcolor{green!25} & \cellcolor{green!25} & & & & \\
        \hline
        Feature-abiding identification & \cellcolor{orange!25} & \cellcolor{orange!25} & \cellcolor{orange!25} & \cellcolor{orange!25} & & & & & & & & \\
        \hline
        Single neuron identification & & & & \cellcolor{orange!25} & \cellcolor{orange!25} & \cellcolor{orange!25} & \cellcolor{orange!25} & & & & & \\
        \hline
        Network topology optimization & \cellcolor{yellow!25} & \cellcolor{yellow!25} & \cellcolor{yellow!25} & & \cellcolor{yellow!25} & \cellcolor{yellow!25} & \cellcolor{yellow!25} & & & & & \\
        \hline
        Validation by simulation & & & \cellcolor{yellow!25} & \cellcolor{yellow!25} & \cellcolor{yellow!25} & & \cellcolor{yellow!25} & \cellcolor{yellow!25} & \cellcolor{yellow!25} & & & \\
        \hline
        Sensitivity analysis & & & & \cellcolor{yellow!25} & \cellcolor{yellow!25} & & & & \cellcolor{yellow!25} & \cellcolor{yellow!25} & \cellcolor{yellow!25} & \cellcolor{yellow!25} mP3 \\
        \hline
    \end{tabular}
\caption{Gantt chart of our proposed project. Planned publication submissions are shown with a short acronym.}    
\label{tab:Gantt}
\end{table}


\section{Associated research partners}\label{sec:partners}

Main research partner to aid this project from the system-identification perspective is Assoc. Prof. \href{https://orcid.org/0000-0003-0005-5223}{Sébastien Court} from University of Innsbruck. He is an expert in data-driven techniques and optimization, which should offer key insight to selection, modification and verification of used methods.

\section{National and/or international cooperation partners}\label{sec:cooperations}

The primary cooperation partner will be \textbf{\href{https://orcid.org/0000-0002-9027-4333}{Lenka Přibylová}}  (the applicant on the Czech Science Foundation side for this project proposal) and the Nonlinear Dynamics Team (see \url{https://science.math.muni.cz/ndteam/}) at Department of Mathematics and Statistics, which is a leader of. The group performs research primarily on applied nonlinear dynamics, constituting a major contributor to the understanding of synchronized coupled neurons and developing bifurcation theories relevant to neuronal behavior for the project.

Moreover, here, we list external specialists accessible through the Nonlinear Dynamics Team and CEITEC MUNI along with their relevance to the proposed project:
\begin{itemize}
    \item \textbf{\href{https://orcid.org/0000-0001-7979-2343}{Milan Brázdil} and his team (Masaryk University):} As a clinical neurologist and head of the First Department of Neurology with vast knowledge and experience in both clinical practice and epilepsy research, Prof. Brázdil's expertise is a key component to validating, from the neurology point of view, the results of identification and optimization problems.
    \item \textbf{\href{https://orcid.org/0000-0002-0232-9518}{Petr Klimeš} (Czech Academy of Sciences):} Expert in neuroengineering and a part of the team from the Institute of Scientific Instruments, CAS, which developed the EEG measurement electrodes and certain analysis techniques, improves the project's validation capabilities in regards to data-driven methods.
    % Can/should I include Hil here?
    \item \textbf{\href{https://orcid.org/0000-0003-1526-3762}{Hil Meijer} (University of Twente):} An associate professor specializing in numerical bifurcation theory and computational neuroscience, Prof. Meijer provides expertise in mathematical modeling of neuronal dynamics via neural mass models.
\end{itemize}

% The primary international cooperation partner will be Dr. Lenka Přibylová, Associate Professor of the Department of Mathematics and Statistics at Masaryk University (MU) -- the applicant on the Czech Science Foundation side for this project proposal. We believe that the combined expertise of both main participating parties is essential for the successful execution of the proposed project. In our opinion, Professor Clason's (University of Graz) extensive knowledge of mathematical programming and related variational methods perfectly complements Associate Professor Lenka Přibylová's (Masaryk University, Brno) proficiency in dynamical systems, bifurcation theory, and neurological modeling. This collaboration, as outlined in the proposed project, is expected to yield unique and significant results. Both international parties will be responsible for their respective parts of the project per areas of their expertise.

% Furthermore, we plan on cooperating with Dr. Sébastien Court from University of Innsbruck regarding data-driven methods of system identification. Collaboration with the rest of the Research and Modeling Team of Nonlinear Dynamics at Dept. of Mathematics and Statistics, MU, lead by Assoc. Prof. Lenka Přibylová, is also intended -- namely with Dr. Veronika Eclerová.

\section{Qualifications of involved researchers}\label{sec:qualifications}

% TODO: Add prof. Clason's qualifications

% TODO: Check & update/reword (I mostly took it from the Deepta's MSCA proposal)


\href{https://orcid.org/0000-0002-9027-4333}{\textbf{Associate Professor Lenka Přibylová}} is an active researcher in nonlinear dynamics and leads the \href{https://science.math.muni.cz/ndteam/}{Nonlinear Dynamics Team} at the Department of Mathematics and Statistics, Faculty of Science, Masaryk University. Her research interests include bifurcation theory, chaos, and the application of nonlinear dynamics in interdisciplinary fields such as neuroscience, biochemistry, population dynamics, and epidemiology. She is an expert in the usage and development of continuation methods in applied nonlinear dynamics, particularly with Matcont software. Her team collaborates with Prof. Hil Meijer from the University of Twente, a co-developer of Matcont, a widely used continuation program for detecting bifurcations.

She leads interdisciplinary research in modeling neuronal dynamics related to epilepsy, collaborating with Milan Brázdil's research team at CEITEC and the Institute of Scientific Instruments of the Czech Academy of Sciences. During the COVID-19 pandemic, she was actively involved in the European COVID-19 Forecast Hub project under the European Centre for Disease Prevention and Control (ECDC). She also participated in a project for Monitoring, Analysis, and Management of Epidemic Situations with the Institute of Health Information and Statistics of the Czech Republic (\href{https://www.uzis.cz/}{ÚZIS}). Additionally, she is a member of the scientific committee of the Center for Modeling of Biological and Social Processes (\href{https://www.bisop.eu/}{BISOP}).

She has successfully collaborated with international teams and researchers from countries including the Netherlands, India, Austria, Ukraine, Russia, and South Africa, applying bifurcation theory. She aims to continue and expand collaborations with both domestic and international researchers in applied nonlinear dynamics, including the following:

\begin{itemize}
    \item Mathematical neuroscience, nonlinear dynamics, numerical methods, and applied bifurcation theory with \href{https://people.utwente.nl/h.g.e.meijer}{Prof. Hil Meijer} at the University of Twente.
    \item Epilepsy related research with \href{https://www.muni.cz/en/about-us/organizational-structure/ceitec/714004-milan-brazdil-rg}{Prof. Brázdil's Research Group}.
    \item Data-driven Koopman methodology with Prof. Sébastien Court from the Department of Mathematics and Digital Science Center at the University of Innsbruck.
    \item Population biology with her former postdoc Deeptajyoti Sen and Prof. Malay Banerjee.
    \item Epidemiology with Prof. Luděk Berec from the University of South Bohemia.
    \item Examining long-term adverse effects of vaccination and COVID-19 with BISOP and ÚZIS.
    \item Notably, through joint doctoral studies of the applicant, a new collaboration with Prof. Christian Clason at the University of Graz will be established, focusing on optimization and machine learning in neuroscience, to develop advanced methods based on specific  characteristics of EEG and other neuronal signals. 
\end{itemize}

She has authored over 20 publications in top-rated international scientific journals, including \textit{Applied Mathematics and Computation}, \textit{Network Neuroscience}, \textit{Communications in Nonlinear Science and Numerical Simulation}, \textit{Journal of Mathematical Biology}, \textit{Journal of Mathematical Analysis and Applications}, \textit{Ecological Complexity}, or \textit{Mathematical Biosciences}.

She has supervised or is currently supervising four Ph.D. students (plus one as a consultant), supervised one postdoctoral researcher under the Marie Skłodowska-Curie Actions Postdoctoral Fellowship, led one ERASMUS+ research internship, and mentored numerous master's and bachelor's students. She is also dedicated to popularization and educational activities for the general public.


TODO the same \textbf{Prof. Christian Clason}


\section{Ethical aspects}\label{sec:ethics}

As this proposal is concerned with fundamental mathematical research, no ethical, safety-related, or regulatory aspects arise.

\section{Gender-related aspects}\label{sec:gender}

As this proposal is concerned with fundamental mathematical research, no sex-specific or gender-related aspects arise.


\newpage

\subsection{List of abbreviations}

\printacronyms[heading=none, display=all]

\newpage
\section*{Annex 1: References}

% TODO: References are too long! (maximum of 5 pages)
\printbibliography[heading=none]

\newpage
\section*{Annex 2: Information on the research institution and requested funding}

% TODO: Add funding!
% Question: Can I be funded mostly by the FWF? It makes much more sense for me financially...
% Also, can travel costs of train Brno-Graz be covered? I'm not sure if I'm eligible for any discounts and one-way ticket is 19 euro... So it would be nice

\newpage
\section*{Annex 3: CVs and descriptions of previous research achievements}

% TODO: Unify style of both CVs and improve the formatting in general
\subsubsection*{Univ.-Prof.\,Dr. Christian Clason}
\label{cv:cclason}

\subsubsection*{Contact details}

University of Graz\\
Department of Mathematics and Scientific Computing\\
Heinrichstrasse 36\\
8010 Graz

\email{c.clason@uni-graz.at}\\
\url{https://homepage.uni-graz.at/de/c.clason}\\
\orcid{0000-0002-9948-8426}

\subsubsection*{Academic milestones and positions}

\enlargethispage{1cm}
\begin{tabularx}{\linewidth}{ll}
    Oct. 2001 -- July 2007 & \emph{Research assistant} \\
    & Centre of Mathematical Sciences, \\&Technische Universität München \\
    Dec. 2006 & \emph{Doctoral Degree in Mathematics}, Technische Universität München\\
    July  2007 -- Mar. 2008 & \emph{Postdoctoral research assistant}  \\
    & Department of Mathematics and Scientific Computing, University of Graz\\
    Apr.  2008 -- Sep. 2009 & \emph{University assistant}  \\
    & Department of Mathematics and Scientific Computing, University of Graz\\
    Oct.  2009 -- June 2013 & \emph{Assistant professor} \\
    & Department of Mathematics and Scientific Computing, University of Graz\\
    Dec. 2012 & \emph{Habilitation in Mathematics}, University Graz\\
    June  2013 -- Jan. 2014 & \emph{Associate professor} \\
    & Department of Mathematics and Scientific Computing, University of Graz\\
    Feb.  2014 -- Feb. 2021  & \emph{Professor (W2) of Inverse Problems}\\
    & Faculty of Mathematics, University Duisburg-Essen\\
    Mar.  2021 -- & \emph{Professor of Mathematical Optimization} \\
    & Department of Mathematics and Scientific Computing, University of Graz \\
\end{tabularx}

\subsubsection*{Main areas of research}

\begin{itemize}
    \item{Nonsmooth PDE-constrained optimization}
    \item{Inverse problems, especially parameter identification in PDEs}
    \item{Optimal control of partial differential equations}
    \item{Applications in medicine and medical imaging}
\end{itemize}


\subsubsection*{Ten most important publications}

\newrefsection[cv/clason.bib]

\nocite{cc:COCV:2009a}
\nocite{cc:MAGMA:2010b}
\nocite{cc:AO:2010a}
\nocite{cc:MRM:2011a}
\nocite{cc:SIIMS:2011a}
\nocite{cc:SICON:2012}
\nocite{cc:IP:2015}
\nocite{cc:JMR:2015}
\nocite{cc:MCRF:2017}
\nocite{cc:SIREV:2021}

\begin{refcontext}[sorting=nyt,labelprefix=CC]
    \printbibliography[heading=none]
\end{refcontext}

\endrefsection

\subsubsection*{Additional research achievements}

\begin{enumerate}
    \item 2015: Winner of \emph{ISMRM Challenge on RF Pulse Design (multi-slice)} (with Christoph Aigner, Armin Rund, Karl Kunisch, Rudolf Stollberger)
    \item since Feb. 2014: \emph{Chair} of \www{https://www.ifip.org/bulletin/bulltcs/memtc07.htm}{IFIP Working Group\,7} on Inverse Problems and Imaging
    \item Associate Editor of \emph{Computational Optimization and Applications} (since 2019), \emph{Journal of Mathematical Analysis and Applications} (since 2017), \emph{Journal of Industrial Management Optimization} (since 2015), \emph{Journal of Inverse and Ill-posed Problems} (since 2014)
    \item since Aug. 2019: \emph{Founding and Managing Editor} of diamond open access journal \www{https://jnsao.episciences.org}{Journal on Nonsmooth Analysis and Optimization}
    \item since Mar. 2021: \emph{Speaker of consortium} \enquote{Optimization and Numerical Analysis}, Doctoral Academy Graz
    \item since September 2024: \emph{Chair} of \www{https://www.ifip.org/bulletin/bulltcs/memtc07.htm}{IFIP Technical Committee 7} on System Modeling and Optimization
    \item Coordinator of Special Research Area (SFB) \www{https://imsc.uni-graz.at/mr-dynamo/}``Mathematics of Reconstruction in Dynamical and Active Models'' Austrian Science Fund (FWF) grant \www{https://dx.doi.org/10.55776/F100800}{10.55776/F100800}, 2025-2029
    \item Supervision of four PhD students (completed)
\end{enumerate}



\textbf{Name}: RNDr. Lenka Přibylová, Ph.D. (previously Baráková) \\
\textbf{Email}: pribylova@math.muni.cz \\
\textbf{ORCID}: \href{https://orcid.org/0000-0002-9027-4333}{0000-0002-9027-4333}\\
\textbf{Research institution}: Department of Mathematics and Statistics, Faculty of Science, Masaryk University, Brno \\
% TODO: Add "..., relevant websites. In addition, a publicly accessible link to the list of all published publications is required"
\textbf{Education}: (degrees, institution, dated; latest to earliest)
\begin{enumerate}[leftmargin=2.5cm]
    \item[1999--2004] Masaryk University, Faculty of Science, doctoral study, mathematical analysis, Ph.D.
    \item[2000] Masaryk University, Faculty of Science, degree RNDr.
    \item[1995--2000] Masaryk University, Fac. of Econ. and Adm., financial business, Bachelor's degree
    \item[1994--1999] Masaryk University, Fac. of Sc., field of study mathematics, Master's degree
    \item[1990--1994] Gymnázium tř. Kap. Jaroše, Brno
\end{enumerate}
\textbf{Positions}: (title, organization, years; latest to earliest)
\begin{enumerate}[leftmargin=2.5cm]
    \item[2023--now] Associate Professor, Applied Mathematics, Dept. of Mathematics and Statistics, Faculty of Science, Masaryk University in Brno
    \item[2006--2023] Assistant Professor, Applied Mathematics, Dept. of Mathematics and Statistics, Faculty of Science, Masaryk University in Brno
    \item[2002--2006] Assistant Professor, Department of Mathematics, Faculty of Agriculture, Mendel University of Agriculture and Forestry in Brno
\end{enumerate}
\textbf{Research interests}: 
\begin{enumerate}[leftmargin=2.5cm]
    \item[\textit{primary}] theory of bifurcations and chaos and application of nonlinear dynamics 
    \item[\textit{secondary}] development of predictive dynamical models
\end{enumerate}
% TODO: Add list of carreer breaks (if any)
% Taken from GAMU proposal
\textbf{Academic publications}:
\begin{itemize}
    \item ECLEROVÁ, V., PŘIBYLOVÁ, L. \& BOTHA, A.E. Embedding nonlinear systems with two or more harmonic phase terms near the Hopf–Hopf bifurcation. Nonlinear Dynamics, Springer Netherlands, 2022.
    \item ŠMÍD, M., BEREC, L., PŘIBYLOVÁ, L., MÁJEK, O., PAVLÍK, T., JARKOVSKÝ, J., WEINER, J., BARUSOVÁ, T. \& TRNKA, J. Protection by Vaccines and Previous Infection Against the Omicron Variant of Severe Acute Respiratory Syndrome Coronavirus 2. The Journal of Infectious Diseases. Oxford University Press, 2022.
    \item BEREC, L., ŠMÍD, M., PŘIBYLOVÁ, L., MÁJEK, O., PAVLÍK, T., JARKOVSKÝ, J., ZAJÍČEK, M., WEINER, J., BARUSOVÁ, T. \& Trnka, J. Protection provided by vaccination, booster doses and previous infection against covid-19 infection, hospitalization or death over time in Czechia. PloS one, 17(7), e0270801, 2022.
    \item HAJNOVÁ, V. and PŘIBYLOVÁ, L. Bifurcation manifolds in predator–prey models computed by Gröbner basis method. Mathematical Biosciences, Elsevier, 2019.
    \item PŘIBYLOVÁ, L. Regime shifts caused by adaptive dynamics in prey–predator models and their relationship with intraspecific competition. Ecological Complexity, 2018
    \item HAJNOVÁ, V. and PŘIBYLOVÁ, L. Two-parameter bifurcations in LPA model. Journal of Mathematical Biology, Springer Verlag, 2017.
    \item PŘIBYLOVÁ, L. \& PENIAŠKOVÁ, A. Foraging facilitation among predators and its impact on the stability of predator–prey dynamics. Ecological Complexity, Amsterdam: ELSEVIER SCIENCE BV, 2017
    \item PŘIBYLOVÁ, L. \& BEREC, L. Predator interference and stability of predator–prey dynamics. Journal of Mathematical Biology, Springer, 2015
    \item PŘIBYLOVÁ, L. Bifurcation routes to chaos in an extended Van der Pol's equation applied to economic models. Electronic Journal of Differential Equations. Texas State University - San Marcos, 2009.
    \item KALAS, J. \& BARÁKOVÁ. Stability and asymptotic behaviour of a two-dimensional differential system with delay. Journal of Mathematical Analysis and Applications. USA: Acad.Press, 2002.
\end{itemize}

% TODO: Add "Additional research achievements" (see page 14 of the guidelines linked above)

% TODO: Add mine CV
\end{document}

