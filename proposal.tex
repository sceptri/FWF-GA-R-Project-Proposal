\documentclass[a4paper,11pt]{scrartcl}
\usepackage{proposal}
\addbibresource{proposal.bib}

% Both acronym and title are subject to change
\sfbacro{OptimizationForFeatures}
\subject{Proposal for International Principal Investigator Project}
\title{Optimization and optimal control of epilepsy-related dynamical systems with respect to signal features}

\begin{document}

\maketitle
\tableofcontents

\newpage

\section{Current state of research}\label{sec:state}

In nature, optimization can be seen as a fundamental principle underlying many diverse mechanisms, from plants orienting themselves toward light as they grow to ants discovering the most efficient paths to resources. It is reasonable, then, to assume that certain observed phenomena are direct outcomes of these underlying optimization processes. This perspective raises an intriguing question: how might applying the concept of optimization to current open problems in science enhance our understanding of them?

Applied mathematics and neuroscience have a long history of complementing each other in the quest to understand the intricacies of the brain. Mathematical models provide a structured way to analyze the dynamics of neuronal systems, offering insights that are difficult to achieve through experimental observation alone. In neuroscience, these models can be used to shed light on various topics such as memory encoding \cite{Rolls1996, Guzman2016}, or information processing, and the generation of rhythmic brain activity \cite{Hoppensteadt1997, Izhikevich2006, Song2018}, which are closely connected to complex synchronization patterns in neuronal networks. 
% TODO: Maybe add applications to other disciplines

Abnormalities in rhythmic brain activity can be attributed to neurological disorders such as epilepsy. From the point of view of neuronal dynamics, these abnormalities manifest themselves as anomalous synchronization within neuronal networks and can lead to spontaneous seizures \cite{Jiruska2013}. Abnormal synchronization is often close tied with the presence of high-frequency oscillations (HFOs), very high-frequency oscillations (VFHOs), and ultra-fast oscillations (UFOs) within depth electroencephalographic (EEG) recordings from patients with focal epilepsy. This presumed connection has fueled the research interest due to the potential use as biomarkers for epileptogenic zones \cite{Jacobs2008, Worrell2011, Staba2011, Jacobs2012, Zijlmans2012, Brazdil2017, Cimbalnik2018, Cimbalnik2020, Brazdil2023}. Nonetheless, taking into account the physiological limitations of action potential firing rates \cite{Augustine2004, Gabbiani2010}, the exact governing mechanism for the generation of such high-frequency signals in neuronal networks remains a mystery \cite{Jiruska2017}.

Existing research endeavors provide insight on how anti-phase synchronization can serve as a potential mechanism for the emergence of HFOs and VHFOs in epileptic EEG signals \cite{Pribylova2024, Sevcik2024, Zathurecky2025}. Specifically, these analyses, based mostly on techniques of bifurcation theory, theoretically examined small numbers of coupled oscillators and provided numerical computations confirming the findings on a large variety of models and scenarios (including stochastic simulations of neuronal populations).
% TODO: Improve this paragraph
% TODO: Add citations to other groups (if relevant)

On the optimization side, optimal control theory (OCT) has proven to be invaluable to control oscillations and network synchrony, as it provides tools for computation of efficient stimulation for linear or non-linear systems \cite{Kirk2004}. In particular, control of neuronal systems has been a target of intensive research \cite{Kao2019, Suppa2016, Liu2018} with closed-loop control and machine learning methods being put the task of optimizing stimulation protocols for the treatment of neurological disorders \cite{Yu2020} and modulating brain activity \cite{Tafazoli2020, Park2019}. Furthermore, models of neuronal populations \cite{Salfenmoser2022, Salfenmoser2024} and their networks \cite{Chouzouris2021} have been the subject of open-loop optimal control techniques. Phase reduction \cite{Zlotnik2012, Dasanayake2011, Pietras2019} techniques were used for control of oscillatory phenomena exhibited by neuronal systems, dramatically simplifying the complex dynamics along stable limit cycles by a phase-only parametrization, allowing even for the study of large-scale networks \cite{Bomela2023}. As a special case, we note that optimal control of oscillators can provide us with tools for a rapid control of systems in time-sensitive manner.

Neuroscience primarily studies complex networks of a large number of intricately coupled neurons, where emergent properties play a key role in governing the dynamics. As such, system identification has been explored as a tool for aiding understanding of systems' structure, be it for identification of the systems as a whole \cite{Brunton2019, Prokop2024}, the discovery of the network configurations or learning missing terms in dynamical systems undergoing bifurcations \cite{VortmeyerKley2021}. A large variety of approaches have been proposed to deal with input noise, robustness \cite{Kaheman2020, Rosafalco2024}, expressive power or utilization of problem-specific properties \cite{Bisheban2017, Jongeneel2022}. Additionally, system identification has been used to enhance other mathematical methods such as optimal control \cite{Morrison2021}, providing invaluable insights into the structure of the systems, which can then be exploited.

Lastly, there has been an active area of research concerning synchronizability of networks depending on their configuration \cite{Hong2004, Nishikawa2003}. Several factors influencing synchronizability have been proposed and studied with substantial efforts going into exploring this phenomenon on small-world networks \cite{Hong2002, Hong2002-1}. Kuramoto model of synchronization has been subject of further inquiries into synchronization \cite{Hong2011}, corresponding optimal networking design targeted at synchronization \cite{Fazlyab2017, Nishikawa2006} and optimal control, which has been used to replicate measured functional relationships from cortical oscillations in human brain \cite{Menara2022}.


\section{Aims of the project}\label{sec:aims}

\section{Innovation}\label{sec:innovation}

\section{Methods}\label{sec:methods}

\section{Work plan and timeline}\label{sec:workplan}

\section{Associated research partners}\label{sec:partners}

\section{National and/or international cooperation partners}\label{sec:cooperations}


\section{Qualifications of involved researchers}\label{sec:qualifications}

\section{Ethical aspects}\label{sec:ethics}

As this proposal is concerned with fundamental mathematical research, no ethical, safety-related, or regulatory aspects arise.

\section{Gender-related aspects}\label{sec:gender}

As this proposal is concerned with fundamental mathematical research, no sex-specific or gender-related aspects arise.


\newpage
\section*{Annex 1: References}

\printbibliography[heading=none]

\newpage
\section*{Annex 2: Information on the research institution and requested funding}

\newpage
\section*{Annex 3: CVs and descriptions of previous research achievements}


\end{document}

