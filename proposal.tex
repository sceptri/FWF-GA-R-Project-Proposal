\documentclass[a4paper,11pt]{scrartcl}
\usepackage{proposal}
\usepackage[usenames,dvipsnames,table]{xcolor}
\usepackage{enumitem}
\usepackage{hyperref}
\addbibresource{proposal.bib}

% Both acronym and title are subject to change
\sfbacro{OptimizationForFeatures}
\subject{Proposal for International Principal Investigator Project}
\title{Optimization and optimal control of epilepsy-related dynamical systems with respect to signal features}

\begin{document}

\maketitle
\tableofcontents

\newpage

\section{Current state of research}\label{sec:state}

% This can certainly be shortened or removed altogether
In nature, optimization can be seen as a fundamental principle underlying many diverse mechanisms, from plants orienting themselves toward light as they grow to ants discovering the most efficient paths to resources. It is reasonable, then, to assume that certain observed phenomena are direct outcomes of these underlying optimization processes. This perspective raises an intriguing question: how might applying the concept of optimization to current open problems in science enhance our understanding of them?

Applied mathematics and neuroscience have a long history of complementing each other in the quest to understand the intricacies of the brain. Mathematical models provide a structured way to analyze the dynamics of neuronal systems, offering insights that are difficult to achieve through experimental observation alone. In neuroscience, these models can be used to shed light on various topics such as memory encoding \cite{Rolls1996, Guzman2016}, or information processing, and the generation of rhythmic brain activity \cite{Hoppensteadt1997, Izhikevich2006, Song2018}, which are closely connected to complex synchronization patterns in neuronal networks. 
% TODO: Maybe add applications to other disciplines

Abnormalities in rhythmic brain activity can be attributed to neurological disorders such as epilepsy. From the point of view of neuronal dynamics, these abnormalities manifest themselves as anomalous synchronization within neuronal networks and can lead to spontaneous seizures \cite{Jiruska2013}. Abnormal synchronization is often close tied with the presence of high-frequency oscillations (HFOs), very high-frequency oscillations (VFHOs), and ultra-fast oscillations (UFOs) within depth electroencephalographic (EEG) recordings from patients with focal epilepsy. This presumed connection has fueled the research interest due to the potential use as biomarkers for epileptogenic zones \cite{Jacobs2008, Worrell2011, Staba2011, Jacobs2012, Zijlmans2012, Brazdil2017, Cimbalnik2018, Cimbalnik2020, Brazdil2023}. Nonetheless, taking into account the physiological limitations of action potential firing rates \cite{Augustine2004, Gabbiani2010}, the exact governing mechanism for the generation of such high-frequency signals in neuronal networks remains a mystery \cite{Jiruska2017}.

Existing research endeavors provide insight on how anti-phase synchronization can serve as a potential mechanism for the emergence of HFOs and VHFOs in epileptic EEG signals \cite{Pribylova2024, Sevcik2024, Zathurecky2025}. Specifically, these analyses, based mostly on techniques of bifurcation theory, theoretically examined small numbers of coupled oscillators and provided numerical computations confirming the findings on a large variety of models and scenarios (including stochastic simulations of neuronal populations).
% TODO: Improve this paragraph
% TODO: Add citations to other groups (if relevant)

On the optimization side, optimal control theory (OCT) has proven to be invaluable to control oscillations and network synchrony \cite{Ruths2014, Orieux2024, Popovych2006, Costa2024}, as it provides tools for computation of efficient stimulation for linear or non-linear systems \cite{Kirk2004}. In particular, control of neuronal systems has been a target of intensive research \cite{Kao2019, Suppa2016, Liu2018} with closed-loop control and machine learning methods being put the task of optimizing stimulation protocols for the treatment of neurological disorders \cite{Yu2020} and modulating brain activity \cite{Tafazoli2020, Park2019}. Furthermore, models of neuronal populations \cite{Salfenmoser2022, Salfenmoser2024} and their networks \cite{Chouzouris2021} have been the subject of open-loop optimal control techniques. Phase reduction \cite{Zlotnik2012, Dasanayake2011, Pietras2019} techniques were used for control of oscillatory phenomena exhibited by neuronal systems, dramatically simplifying the complex dynamics along stable limit cycles by a phase-only parametrization, allowing even for the study of large-scale networks \cite{Bomela2023}. As a special case, we note that optimal control of oscillators can provide us with tools for a rapid control of systems in time-sensitive manner \cite{Kamzolkin2024, Santaniello2011}.

Neuroscience primarily studies complex networks of a large number of intricately coupled neurons, where emergent properties play a key role in governing the dynamics. As such, system identification has been explored as a tool for aiding understanding of systems' structure, be it for identification of the systems as a whole \cite{Brunton2019, Prokop2024}, the discovery of the network configurations \cite{Owens2022} or learning missing terms in dynamical systems undergoing bifurcations \cite{VortmeyerKley2021}. A large variety of approaches have been proposed to deal with input noise, robustness \cite{Kaheman2020, Rosafalco2024}, expressive power or utilization of problem-specific properties \cite{Bisheban2017, Jongeneel2022}. Additionally, system identification has been used to enhance other mathematical methods such as optimal control \cite{Morrison2021, Taylor2015}, providing invaluable insights into the structure of the systems, which can then be exploited.

Lastly, there has been an active area of research concerning synchronizability of networks depending on their configuration \cite{Hong2004, Nishikawa2003}. Several factors influencing synchronizability have been proposed and studied with substantial efforts going into exploring this phenomenon on small-world networks \cite{Hong2002, Hong2002-1}. Kuramoto model of synchronization \cite{Kuramoto1984} has been subject of further inquiries into synchronization \cite{Hong2011}, corresponding optimal networking design targeted at synchronization \cite{Fazlyab2017, Nishikawa2006} and optimal control, which has been used to replicate measured functional relationships from cortical oscillations in human brain \cite{Menara2022}.

% TODO: Maybe add something about signal features? like \cite{Wedler2022, Srivastava2016, Perlin2016}

\section{Aims of the project}\label{sec:aims}

Our principal desire for this project is to connect the bifurcation theory and the optimization viewpoints to neuroscience with special emphasis on focal epilepsy. This connection should reveal currently hidden insights into the governing principles of dynamics of neuronal networks.

In particular, we aim to utilize the techniques of optimal network design \cite{Motter2007} and optimal control \cite{Baggio2021} to achieve synchronous dynamics with characteristics related to the epileptogenic zone (under physiological constraints). As such, this should closely align with modeling of a local field potential using coupled networks, and performing certain optimization procedures with respect to coupling strengths or delays between oscillators. 

In parallel, we strive to set up a mathematically rigorous notion of various signal features (e.g. frequency bands for spiking or bursting, interburst intervals, VHFOs, and UFOs, etc.) \cite{Wedler2022, Srivastava2016} to be used in optimization or epilepsy-event detection, etc. Together with modifications of the current optimization and optimal control methodologies to accommodate for requirements of certain signal features, such construction should allow for selection or definition of a more application-appropriate cost function, aiding the potential impact of the research.

Our third goal lies mostly in system identification, where we wish to study derivative-free methods \cite{Bortz2023, Lejarza2022, Messenger2024}. Specifically, we will attempt identification of a network configuration from in vivo, in vitro and in silico experimental data \cite{Champion2019}. This will enhance our understanding of epileptic tissue and, ideally, improve the localization of the epileptogenic zone based on interictal biomarkers (HFOs, VHFOs and UFOs). Note that we would also like to explore the connection of optimal control and system identification. Importantly, the identification task should also be focused on discovering models or networks whose solution do not necessarily fit the experiment data exactly, but contain the same signal features.

\section{Innovation}\label{sec:innovation}

We perceive the key contributions to a broader scientific community in several areas. Firstly, the rigorous treatment of signal features should be relevant and highly useful not only in the field of neuroscience, but also among other fields where certain properties of an output signal of a controlled system are desired. Applications for a given system include detection of events, modeling thereof, and optimal control such that certain requirements on the features of the output signal are fulfilled.

Crucial innovation of the proposed project lies in the application of the notion of signal features specifically to mathematical treatment of HFOs, VHFOs and even UFOs. We believe such approach can greatly enhance current modeling capabilities in this niche, providing sound explanation for the emergence of non-physiological frequencies in local mean field potential measurements and their importance as biomarkers. Moreover, studying ways of controlling the anomalous oscillations may likely prove fruitful in their suppression. By further exploring system identification from the complex EEG signals, possibly also constrained by certain signal features, it is plausible we will be able to exploit discovered structures in our benefit.

% TODO: Maybe, add something about SINDyOC and derivative-free equation discovery methods (and OCT <-> eq. discovery relation)

\section{Methods}\label{sec:methods}

% TODO: Find relevant citations (and be more specific) 
There are several ways one could go about mathematically defining the concept of a signal feature. The most straightforward way is probably choosing a reference function, which should ideally capture the essence of our signal feature and a similarity metric, for example an inner product in a Hilbert space (inducing a correlation between the reference and input signals). Such idea should be compatible with the theory of wavelet transforms \cite{Gupta2005, Ganjalizadeh2022, Dirkx2023, Brochard2022, Rocha2011}, which incorporates the notion translations and scaling \cite{Mallat2009} -- two very important properties any reasonable signal feature definition must allow.

Another possible approach we have considered are techniques of topological data analysis \cite{Edelsbrunner2010}, namely the usage of topological invariants or persistent homologies \cite{Carriere2021, Nigmetov2024, Myers2022}. Although this approach requires far more theoretical background, it should possibly allow us to express a bigger collection of signal features in mathematically rigorous way. Moreover, either of these two methods can likely succeed each other and thus mitigate the risk of either being unsuitable.

We aim to perform optimal control using standard tools of calculus variations, like the Lagrange multiplier approach \cite{Kunisch2008}. Nonetheless, for the high dimensional case of optimal control of a neuronal network (with each node of the network being possibly a relatively high dimensional model of neuron itself), we may need to turn to more novel methods, such as geometric optimal control \cite{Jurdjevic1996, Agrachev2004}, taking advantage of the topology of the network \cite{Gao2022, Nishikawa2006, Gueant2019}, or exploiting the turnpike property \cite{Zaslavski2023, Gugat2024}.

In terms of mathematical programming, convex optimization \cite{Boyd2004} and corresponding relaxations \cite{Eltved2021} seem favorable. For practical purposes, considerations regarding robust optimization \cite{Ben-Tal2009} will likely be necessary, as well as careful selection of numerical optimization methods, for example, (adaptive) Barzilai-Borwein method \cite{Zhou2024}. Network topology optimization primarily relies on combinatorial optimization \cite{Korte2005, Dai2011, Osmolovskii2023}, discrete calculus \cite{Grady2010}, and graph theory results.

Let us also note the EEG data \cite{Brazdil2017} were measured using hippocampal depth electrode implanted within the presurgical investigation of patients suffering from drug-resistant focal epilepsy. The microelectrodes and macroelectrodes were spatially distributed, and the measured neural mass activities at different places were interconnected in the brain. This introduces the need to study and analyze the dynamics of the subsystems (subpopulations) of neural mass with various interconnections.

\section{Work plan and timeline}\label{sec:workplan}

% Project description must include:
% - The distribution of work between the partners
% - The added value of the international co-operation

We divide our proposed project into 5 tasks spread out over the three year duration, see Table \ref{tab:Gantt} for a Gannt chart of planned progress:

\begin{itemize}
    \item[\textbf{Task 1.}] \textit{(blue)} Definition of key signal features (for example, frequency bands for spiking or bursting, VFHOs or UFOs, interburst intervals) via wavelet theory or topologically. Development of event detection techniques based on chosen methodology. Application of signal feature detection on real-world experimental data \cite{Brazdil2017}, building upon data analysis of \cite{Halastova2025} in collaboration with Veronika Eclerová, Ph.D., from Masaryk University, Brno.
    
    \item[\textbf{Task 2.}] \textit{(purple)} Construction of loss functions corresponding to selected signal features for both optimization and optimal control scenarios. Building analytical and numerical methods for solving proposed problems, including, but not limited to, considerations of discretization, computational complexity, and robustness. Proof-of-concept implementation and numerical experiments.

    \item[\textbf{Task 3.}] \textit{(green)} Local field potential modeling using a coupled network. Optimization and optimal network topology design of such system with respect to coupling strengths (i.e. the network configuration and topology) and delays \cite{Ritschel2024, Boccia2016}. In particular, optimization-based modeling of VHFOs and UFOs.

    \item[\textbf{Task 4.}] \textit{(orange)} Study of derivative-free system identification methods, mainly based on \textit{sparse identification of nonlinear dynamics} (SINDy) \cite{Brunton2016}, in collaboration with Dr. Sébastien Court from Innsbruck. Formulation of the task as an optimal control problem. Derivation of optimality conditions with adjoint method. Numerical experiments and comparison to SINDy and its various modifications. Incorporation of signal features and corresponding loss functions to the system identification framework will be examined.

    \item[\textbf{Task 5.}] \textit{(yellow)} Parameter and network topology identification from \textit{in vivo} (see \textit{Task 1}), \textit{in vitro}, and \textit{in silico} data. Both theoretical analysis and numerical schemes will be considered, along with development of suitable numerical experiments. Our goal will be to discover models and networks, whose solution do not necessarily fit the experiment data exactly, but contain the same signal features.
\end{itemize}
% TODO: Add anything non-method related (any visits/conferences?)

% TODO: Improve wording, I'm afraid
We aim to co-author a paper (coP1) on derivative-free system identification, inspired by SINDy, in collaboration with Dr. Sébastien Court from the University of Innsbruck. Additionally, we plan to apply event detection methods based on our signal features to real-world EEG measurements, contributing to a data analysis paper (coP2) in partnership with Veronika Eclerová, Ph.D., from Masaryk University in Brno.

On a similar note, we expect to produce a publication on the entire framework of signal feature definition and subsequent optimization/optimal control (mP3). We expect the application of this methodology to explain to emergence of VHFOs, and UFOs in neuronal networks to warrant its own publication in a more neuroscience-oriented journal (aP4). Lastly, we believe optimization and discovery of neuronal networks in conjunction with signal features and novel system identification methods will offer enough original and substantial results to justify a separate article (mP4).

\setlength\extrarowheight{3pt}
\begin{table}[h!]
    \centering
     \addtolength{\tabcolsep}{-3pt}
     \fontsize{10pt}{11pt}\selectfont
    \begin{tabular}{|c|c|c|c|c|c|c|c|c|c|c|c|c|}
    \hline
    \multicolumn{13}{|c|}{\cellcolor{gray!50} Year 1} \\
        \hline
        Work plan item & 1 & 2 & 3 & 4 & 5 & 6 & 7 & 8 & 9 & 10 & 11 & 12 \\
        \hline
        Derivative-free SINDy & \cellcolor{orange!25} & \cellcolor{orange!25} & \cellcolor{orange!25} & \cellcolor{orange!25} coP1 & & & & & & & & \\
        \hline
        Signal feature definition & & & & \cellcolor{blue!25} & \cellcolor{blue!25} & \cellcolor{blue!25} & \cellcolor{blue!25} & & & & & \\
        \hline
        Event detection & & & & & & \cellcolor{blue!25} & \cellcolor{blue!25} & \cellcolor{blue!25} & & & & \\
        \hline
        Event signal analysis & & & & & & & \cellcolor{blue!25} & \cellcolor{blue!25} & \cellcolor{blue!25} & \cellcolor{blue!25} coP2 & & \\
        \hline
        Loss functions & & & & & & & & & & \cellcolor{purple!25} & \cellcolor{purple!25} & \cellcolor{purple!25} \\
        
    \hline
    \multicolumn{13}{|c|}{\cellcolor{gray!50}Year 2} \\
        \hline
        Work plan item & 13 & 14 & 15 & 16 & 17 & 18 & 19 & 20 & 21 & 22 & 23 & 24\\
        \hline
        Loss functions & \cellcolor{purple!25} & \cellcolor{purple!25} & & & & & & & & & & \\
        \hline
        Analytical methods & \cellcolor{purple!25} & \cellcolor{purple!25} & \cellcolor{purple!25} & \cellcolor{purple!25} & \cellcolor{purple!25} & & & & & & & \\
        \hline
        Numerical methods & & & & \cellcolor{purple!25} & \cellcolor{purple!25} & \cellcolor{purple!25} & \cellcolor{purple!25} & & & & & \\
        \hline
        Proof-of-concept & & & & & & \cellcolor{purple!25} & \cellcolor{purple!25} & \cellcolor{purple!25} mP3 & & & & \\
        \hline
        Local field potential optim. & & & & & & & & \cellcolor{green!25} & \cellcolor{green!25} & \cellcolor{green!25} & & \\
        \hline
        VHFOs, UFOs modeling & & & & & & & & & & \cellcolor{green!25} & \cellcolor{green!25} & \cellcolor{green!25}\\
    \hline  
    \multicolumn{13}{|c|}{\cellcolor{gray!50}Year 3} \\
        \hline
        Work plan item & 25 & 26 & 27 & 28 & 29 & 30 & 31 & 32 & 33 & 34 & 35 & 36\\
        \hline
        VHFOs, UFOs modeling & \cellcolor{green!25} aP4 & & & & & & & & & & & \\
        \hline
        Feature-abiding identification & \cellcolor{orange!25} & \cellcolor{orange!25} & \cellcolor{orange!25} & \cellcolor{orange!25} & & & \cellcolor{orange!25} & \cellcolor{orange!25} & \cellcolor{orange!25} & \cellcolor{orange!25} & & \\
        \hline
        Fixed topology -- analysis & & & \cellcolor{yellow!25} & \cellcolor{yellow!25}  & & & & & & & & \\
        \hline
        Fixed topology -- numerics & & & & \cellcolor{yellow!25} & \cellcolor{yellow!25} & \cellcolor{yellow!25} & & & & & & \\
        \hline
        Network optim. -- analysis & & & & & & & \cellcolor{yellow!25} & \cellcolor{yellow!25} & \cellcolor{yellow!25} & \cellcolor{yellow!25} & & \\
        \hline
        Network optim. -- numerics & & & & & & & & & \cellcolor{yellow!25} & \cellcolor{yellow!25} & \cellcolor{yellow!25} & \cellcolor{yellow!25} mP4 \\
        \hline
    \end{tabular}
\caption{Gantt chart of our proposed project. Planned publications are shown with a short acronym.}    
\label{tab:Gantt}
\end{table}


\section{Associated research partners}\label{sec:partners}

% TODO: Nobody or Dr. Sébastien Court?

\section{National and/or international cooperation partners}\label{sec:cooperations}

% Should assoc. prof. Lenka Přibylová and her team be even included here?

The primary international cooperation partner will be Dr. Lenka Přibylová, Associate Professor of the Department of Mathematics and Statistics at Masaryk University (MU) -- the applicant on the Czech Science Foundation side for this project proposal. We believe that the combined expertise of both main participating parties is essential for the successful execution of the proposed project. In our opinion, Professor Clason's (University of Graz) extensive knowledge of mathematical programming and related variational methods perfectly complements Associate Professor Lenka Přibylová's (Masaryk University, Brno) proficiency in dynamical systems, bifurcation theory, and neurological modeling. This collaboration, as outlined in the proposed project, is expected to yield unique and significant results. Both international parties will be responsible for their respective parts of the project per areas of their expertise.

Furthermore, we plan on cooperating with Dr. Sébastien Court from University of Innsbruck regarding data-driven methods of system identification. Collaboration with the rest of the Research and Modeling Team of Nonlinear Dynamics at Dept. of Mathematics and Statistics, MU, lead by Assoc. Prof. Lenka Přibylová, is also intended -- namely with Dr. Veronika Eclerová.

% TODO: Should we also include Hil Meijer?

\section{Qualifications of involved researchers}\label{sec:qualifications}

% TODO: Add prof. Clason's qualifications

% TODO: Check & update/reword (I mostly took it from the Deepta's MSCA proposal)
Associate Professor Lenka Přibylová is an active researcher in nonlinear dynamics and leading the nonlinear dynamics group at that department. She has a strong interest in bifurcation theory and chaos, application of nonlinear dynamics in interdisciplinary fields like biochemistry, population dynamics, neuroscience, and epidemiology. She is an expert in Matcont continuation software and continuation methods usage and development in applied nonlinear dynamics, her team is collaborating with prof. Hil Meijer in University of Twente who is one of Matcont developers and authors of this worldwide used continuation program for detecting bifurcations. She has been diligently involved in the scientific committee of the Center for Modeling of Biological Science and Social Processes and is a member of the European Covid-19 forecast hub. She participated in project for Monitoring, Analysis and Management of Epidemic Situations with The Institute of Health Information and Statistics of the Czech Republic (ÚZIS). She has successfully collaborated with researchers in the Czech Republic, Ukraine, Russia, and South Africa in applied bifurcation theory and expects to continue and extend collaboration with domestic and international researchers towards applied nonlinear dynamics: in data-driven Koopman methodology with Sebastien Court from Department of Mathematics \& Digital Science Center at University of Innsbruck, in Matcont, numerical methods and applied bifurcation theory for usage in neuroscience with Hil Meijer at the University of Twente, and population biology and epidemiology with Luděk Berec from the Czech Academy of Sciences, with the Center for Modeling of Biological Science and Social Processes, ECDC Forecast Hub researchers’ team, and The Institute of Health Information and Statistics of the Czech Republic (ÚZIS). She has more than 20 publications, including international journals (Journal of Mathematical Biology, Journal of Mathematical Analysis and Application, Ecological Complexity, Mathematical Biosciences) and conference proceedings. She has supervised or has under supervision 4 Ph.D. students (plus one as a consultant), leaded one postdoc in ERASMUS+ research internship, as well as many masters and bachelor students. She is very dedicated to popularization and educational activities for the broad public.

\section{Ethical aspects}\label{sec:ethics}

As this proposal is concerned with fundamental mathematical research, no ethical, safety-related, or regulatory aspects arise.

\section{Gender-related aspects}\label{sec:gender}

As this proposal is concerned with fundamental mathematical research, no sex-specific or gender-related aspects arise.


\newpage
\section*{Annex 1: References}

% TODO: References are too long! (maximum of 5 pages)
\printbibliography[heading=none]

\newpage
\section*{Annex 2: Information on the research institution and requested funding}


% TODO: Add funding!
% Question: Can I be funded mostly by the FWF? It makes much more sense for me financially...
% Also, can travel costs of train Brno-Graz be covered? I'm not sure if I'm eligible for any discounts and one-way ticket is 19 euro... So it would be nice

\newpage
\section*{Annex 3: CVs and descriptions of previous research achievements}

% TODO: Add Clason's CV

% TODO: Maybe improve formatting
% See for more information: https://www.fwf.ac.at/fileadmin/Website/Dokumente/Foerdern/Portfolio/Einzelprojekte/fwf_pat_application_guidelines.pdf?page=13

\textbf{Name}: RNDr. Lenka Přibylová, Ph.D. (previously Baráková) \\
\textbf{Email}: pribylova@math.muni.cz \\
\textbf{ORCID}: \href{https://orcid.org/0000-0002-9027-4333}{0000-0002-9027-4333}\\
\textbf{Research institution}: Department of Mathematics and Statistics, Faculty of Science, Masaryk University, Brno \\
% TODO: Add "..., relevant websites. In addition, a publicly accessible link to the list of all published publications is required"
\textbf{Education}: (degrees, institution, dated; latest to earliest)
\begin{enumerate}[leftmargin=2.5cm]
    \item[1999–2004] Masaryk University, Faculty of Science, doctoral study, mathematical analysis, Ph.D.
    \item[2000] Masaryk University, Faculty of Science, degree RNDr.
    \item[1995–2000] Masaryk University, Fac. of Econ. and Adm., financial business, Bachelor's degree
    \item[1994–1999] Masaryk University, Fac. of Sc., field of study mathematics, Master's degree
    \item[1990–1994] Gymnázium tř. Kap. Jaroše, Brno
\end{enumerate}
\textbf{Positions}: (title, organization, years; latest to earliest)
\begin{enumerate}[leftmargin=2.5cm]
    \item[2023–now] Associate Professor, Applied Mathematics, Dept. of Mathematics and Statistics, Faculty of Science, Masaryk University in Brno
    \item[2006–2023] Assistant Professor, Applied Mathematics, Dept. of Mathematics and Statistics, Faculty of Science, Masaryk University in Brno
    \item[2002–2006] Assistant Professor, Department of Mathematics, Faculty of Agriculture, Mendel University of Agriculture and Forestry in Brno
\end{enumerate}
\textbf{Research interests}: 
\begin{enumerate}[leftmargin=2.5cm]
    \item[\textit{primary}] theory of bifurcations and chaos and application of nonlinear dynamics 
    \item[\textit{secondary}] development of predictive dynamical models
\end{enumerate}
% TODO: Add list of carreer breaks (if any)
% Taken from GAMU proposal
\textbf{Academic publications}:
\begin{itemize}
    \item ECLEROVÁ, V., PŘIBYLOVÁ, L. \& BOTHA, A.E. Embedding nonlinear systems with two or more harmonic phase terms near the Hopf–Hopf bifurcation. Nonlinear Dynamics, Springer Netherlands, 2022.
    \item ŠMÍD, M., BEREC, L., PŘIBYLOVÁ, L., MÁJEK, O., PAVLÍK, T., JARKOVSKÝ, J., WEINER, J., BARUSOVÁ, T. \& TRNKA, J. Protection by Vaccines and Previous Infection Against the Omicron Variant of Severe Acute Respiratory Syndrome Coronavirus 2. The Journal of Infectious Diseases. Oxford University Press, 2022.
    \item BEREC, L., ŠMÍD, M., PŘIBYLOVÁ, L., MÁJEK, O., PAVLÍK, T., JARKOVSKÝ, J., ZAJÍČEK, M., WEINER, J., BARUSOVÁ, T. \& Trnka, J. Protection provided by vaccination, booster doses and previous infection against covid-19 infection, hospitalization or death over time in Czechia. PloS one, 17(7), e0270801, 2022.
    \item HAJNOVÁ, V. and PŘIBYLOVÁ, L. Bifurcation manifolds in predator–prey models computed by Gröbner basis method. Mathematical Biosciences, Elsevier, 2019.
    \item PŘIBYLOVÁ, L. Regime shifts caused by adaptive dynamics in prey–predator models and their relationship with intraspecific competition. Ecological Complexity, 2018
    \item HAJNOVÁ, V. and PŘIBYLOVÁ, L. Two-parameter bifurcations in LPA model. Journal of Mathematical Biology, Springer Verlag, 2017.
    \item PŘIBYLOVÁ, L. \& PENIAŠKOVÁ, A. Foraging facilitation among predators and its impact on the stability of predator–prey dynamics. Ecological Complexity, Amsterdam: ELSEVIER SCIENCE BV, 2017
    \item PŘIBYLOVÁ, L. \& BEREC, L. Predator interference and stability of predator–prey dynamics. Journal of Mathematical Biology, Springer, 2015
    \item PŘIBYLOVÁ, L. Bifurcation routes to chaos in an extended Van der Pol's equation applied to economic models. Electronic Journal of Differential Equations. Texas State University - San Marcos, 2009.
    \item KALAS, J. \& BARÁKOVÁ. Stability and asymptotic behaviour of a two-dimensional differential system with delay. Journal of Mathematical Analysis and Applications. USA: Acad.Press, 2002.
\end{itemize}

% TODO: Add "Additional research achievements" (see page 14 of the guidelines linked above)

\end{document}

